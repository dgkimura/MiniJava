\documentclass[letterpaper]{article}
\usepackage{fullpage}

\title{CSE 401 Final Report}
\author{David Kimura \and David Koenig}

\begin{document}

\maketitle

\section{What works}

The scanner and parser should work without any problems. Type information
is extracted from the tree. The type checker mostly works but breaks on some
of our test cases; it is thus not called when the \texttt{MiniJava} class is
run.

All language features are implemented. Code generation is functional, and
all test programs work. The generated assembly is somewhat unwieldy in
places but should function.

We took liberties with Java in places where we were allowed to do so. We
made no effort to ensure that a method's invocant is evaluated before its
parameters. We did not bother to check whether an array assignment should
evaluate the index before the new value or vice versa. Arguments to methods
are evaluated right to left.

Although MiniJava specifies that C-style comments should nest, we follow the
behavior of Java and stop processing comments after the first \texttt{*/}.

\section{What does not work}

The type verifier still has some problems. In the \texttt{MiniJava} we have
disabled it for the time being. Generally, valid programs will compile, and
invalid programs will cause the compiler to throw all manner of exceptions,
usually complaining about null pointers.

\section{Test programs tried}

All eight provided test examples compile and run as expected; their output
has been verified to be the same as what Oracle's Java produces. These cover
functionality both simple (arithmetic, printing, et cetera) and advanced
(arrays and vtables). These live in the
\texttt{SamplePrograms/SampleMiniJavaPrograms} directory of our source tree.

\section{Additional features}

We initially had planned to add support for floating point arithmetic, but
this never came to fruition. The visitor classes have support, however, so
here is what we would do to get it working:

\begin{itemize}

\item Add support to the scanner. We would likely want to add new builtins
to perform rounding and calculating the floor or ceiling of a floating point
number.

\item Add support to the parser. We would probably create a new nonterminal
`number' which would derive either an integer or a floating point number.

\item Modify the type checker to allow floating point numbers as operands to
arithmetic operators. We would also need to modify the tree to handle
promoting an integer to a floating point as required.

\item Add support for generation of operations. We would probably target the
SSE instruction families instead of x87 for simplicity.

\end{itemize}

\section{Division of work}

The scraper was done mostly by Koenig (although he intended to have Kimura
work on it, he got into the zone and accidentally finished it in one
sitting). Refinement and occasional fixes were contributed by Kimura.

The parser was split about halfway between Koenig and Kimura.

Koenig completed the pass that pulls type information from the AST, with
minor refinements from Kimura.

Kimura completed the pass that actually checked the type information, with
minor refinements from Kimura. This was not entirely functional at time of
turn in, so it has been disabled, but it should mostly work.

Kimura worked on the vtable implementation in the code generator, as well as
various random pieces of it. Koenig worked on the implementation of variable
scope, as well as the rest of the various random pieces.

\end{document}

